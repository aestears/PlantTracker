\documentclass[12pt, letterpaper]{article}
\usepackage[utf8]{inputenc}
\usepackage[english]{babel}
\usepackage{fancyhdr}
\usepackage{geometry}
%\usepackage{natbib}
\usepackage{indentfirst} % to indent the first paragraph after a section heading
\usepackage{apacite}
\bibliographystyle{apacite-copy}
%\setcitestyle{aysep={,}} % separate author name and year with a comma
\renewcommand{\BRetrievedFrom}{" "}
\renewcommand{\BRetrieved}{" "}
\renewcommand{\doiprefix}{https://doi.org/}

\usepackage{booktabs, tabularx, longtable}
\usepackage{array}
\usepackage{colortbl}
\usepackage{graphicx} 
\usepackage[section]{placeins}
\usepackage{amsmath}
\usepackage[leftcaption]{sidecap}
\graphicspath{ {./figures/} }
\usepackage{setspace}
\usepackage{caption}
\captionsetup[table]{font={stretch=1.05, small}}   %% change table caption spacing
\captionsetup[figure]{font={stretch=1.05, footnotesize}} %% change figure caption spacing
\usepackage{titlesec}
\titleformat{\section}
 {\normalfont\fontsize{14}{15}\bfseries}{\thesection}{1em}{} %change the size of section header font
\titleformat{\subsection}
 {\normalfont\fontsize{12}{15}\bfseries}{\thesubsection}{1em}{} %change the size of subsection header font
\renewcommand{\thesection}{} %remove the section numbers
\renewcommand{\thesubsection}{} %remove subsection numbers

%try to deal with the problem of too long figure caption
\DeclareCaptionLabelFormat{adja-page}{\hrulefill\\#1 #2 \emph{(previous page)}}
\usepackage{subfig}

%try to deal with issue of figures going to the end of the document
\makeatletter
\AtBeginDocument{%
 \expandafter\renewcommand\expandafter\subsection\expandafter{%
  \expandafter\@fb@secFB\subsection
 }
}
\makeatother

\geometry{margin = 1in}
\pagestyle{fancy}
\fancyhf{}
%set the header
\rhead{\thepage}
\lhead{A. Stears}
%set the line spacing
\renewcommand{\baselinestretch}{2}
%add line numbers
\usepackage{lineno}
\linenumbers

\begin{document}

\begin{flushleft}
\Large{\textbf{plantTrackeR: An R package to translate plant quadrat maps into demographic data}} 

\normalsize{Alice E. Stears$^{1*}$, Peter B. Adler$^2$, Shannon Albeke$^{3}$, Jared Studyvin$^{4}
$, Daniel C. Laughlin$^1$}

\small{$^1$Botany Department and Program in Ecology, University of Wyoming, Laramie, WY; \linebreak
$^2$Department of Wildland Resources and the Ecology Center, Utah State University, Logan, UT; \linebreak
$^3$Wyoming Geographic Information Science Center, University of Wyoming,
Laramie, WY; \linebreak
$^4$ Department of Mathematics and Statistics, University of Wyoming, Laramie, WY \linebreak 
}
\small{$^*$Corresponding Author: astears@uwyo.edu}

\end{flushleft}

\section{Abstract}
\section{Introduction}
Long-term observations of demographic processes such as growth and survival in biological organisms are both rare and valuable for answering many pressing ecological questions. Demographic rates fundamentally determine fitness, so any efforts to determine drivers of organism fitness should address, if not explicitly include, an analysis of demographic rates. For example, the most successful conservation efforts depend on a clear understanding of how different demographic rates are impacted by the environment, and must also quantify the relative importance of those demographic rates for determining fitness for the organism in question (citation). 

However, collecting true demographic data to calculate demographic rates is an involved process. Individual organisms must be tagged with a unique identifying marker, mapped, and measured along multiple dimensions every year for a minimum of three years. Three consecutive years of data is the absolute minimum, however, and the accuracy and utility of demographic data typically increases with the number of consecutive years of data collection. Despite their importance, long-term datasets are rare, and long-term demographic datasets are practically non-existent. Complete demographic data collection is time-consuming and tedious even in the short-term, and lack of funding, labor, permits, institutional interest, and an inconsistent methodology all complicate or more likely condemn long-term demographic data collection. Because of this, the availability of demographic data is inversely related to its importance.  

One potential source of demographic information lies in historical datasets. A particular subset of these historical datasets were collected using a method called the 'chart quadrat', which was a formalized protocol for using quadrats, or rectangular frames laid down around vegetation, to track plant individuals, populations, and communities through time \cite{Clements1907}. The ‘chart-quadrat,’ a permanent, 1 m$^2$ within which the basal cover and species identity of every individual plant is mapped \cite{Hill1920}. Each individual is identified and mapped as a polygon, unless it  has a negligible basal area (i.e. a forb that has a single stem), in which case it is mapped as a point. Repeat sampling of chart-quadrats over time generates maps that show how overall ground cover, relative species cover, and distribution of individuals has shifted over the sampling period. Chart-quadrat data has been used to study change in plant communities over time, but also to determine demographic patterns of individual species \cite{Albertson1965, Wright1976}.

Use of chart-quadrats became standard practice for vegetation monitoring in the early to mid-20$^{th}$ century, particularly in rangeland management \cite{Albertson1965, Hill1920}. Scientists established (and are still establishing) hundreds of chart-quadrats at many rangeland research sites throughout western North America \cite{Zachmann2010, Chu2013, Anderson2012, Anderson2011, Adler2007}. Because two corners of each chart-quadrat are marked with pieces of angle-iron, it has also been possible to re-locate many quadrats that were established as early as 1912, making it conceivable to extend these long-term datasets even farther \cite{Dowling2015, Adler2019}. Some historic chart-quadrat maps have been lost, but many have been recovered, digitized into shapefiles and are being used to test ecological theory. At this point, chart-quadrat maps have been digitized for seven sites, each of which have between 24 and 178 quadrats: Fort Hays State University Pasture in Hays, KS; Fort Keogh Experimental Range in Miles City, MT; the Central Plains Experimental Range in Nunn, CO; the US Sheep Experiment Station in Dubois, ID; the Coconino National Forest in Flagstaff, AZ; the Santa Rita Experimental Range near Tucson, AZ; and the Jornada Experimental Range near Las Cruces, NM \cite{Adler2007, Anderson2011, Zachmann2010, Chu2013, Anderson2012, Christensen2021Quadratbased19152016}. 

Although these chart-quadrat datasets were not sampled with a methodology consistent with traditional demographic studies (i.e. each individual plant is given a unique identification and tagged so that it can be followed through multiple transitions), the spatially-explicit nature of these map datasets means that we can extract demographic data such as individual rates of growth and survival, and plot-level rates of reproduction (via recruitment of seedlings). If an individual plant of the same species occurs at approximately the same location from year to year, it can be reasonably assumed that it is the same individual. In this manner, we can determine individual growth and survival. This process of tracking un-marked, mapped individuals to generate demographic data works well for chart-quadrat maps, but can also be used in any scenario where plants--or any stationary organism--are mapped annually. Transforming maps in this way to generate demographic datasets opens up many exciting possibilities. Perhaps most importantly in the case of the chart-quadrat data, we now are able to ask questions about plant demographic processes over very long time scales. At the Jornada Experimental Range, quadrats were mapped for 23 consecutive years, giving us growth and survival rates for 22 annual transitions. A dataset of this duration would be nearly impossible to collect using traditional demographic collection methods. Demographic data extracted from chart-quadrat maps or other historical datasets also offers an excellent opportunity to determine how growth, survival, and recruitment have shifted in response to environmental change. Three chart-quadrat locations were sampled as early as 1915, and six sites have data from earlier than 1950. 

Here, we describe the software package 'plantTracker', which is an add-on for R statistical software \cite{RCoreTeam2021} that is available for download through the Comprehensive R Archive Network (CRAN) and GitHub. plantTracker contains functions to extract growth, survival, and recruitment information from digitized shapefiles of quadrat maps. This package provides users with a straightforward set of tools to fully leverage the capabilities of quadrat map data. 

\section{plantTracker}
plantTracker depends on the "sf" \cite{Pebesma2018}, "Matrix" \cite{Bates2019Matrix:Methods}, and "igraph" \cite{Csardi2006TheResearch} R add-on packages, and contains two primary "workhorse" functions. The first, trackSpp(), identifies individuals through time by comparing sequential quadrat maps, and generates growth and survival information for each individual in each year. The second function, getNeighbors() calculates a metric of either conspecific or heterospecific competition for each individual in each year, which is either the number of individuals within a user-specified buffer around the focal plant, or the proportion of that buffer area that is occupied by other plants. plantTracker also has functions to prepare data for analysis and generate some additional demographic metrics. The main functions require two inputs: 1) a spatial data.frame in the 'sf' format \cite{Pebesma2018}, where each row contains polygon data for one observation in a given year, and 2) a named list where each element is named after a quadrat in the map dataset and contains a vector of the years in which that quadrat was sampled. The main functions return an "sf" data.frame in which each row contains polygon data for one individual in a given year, as well as columns providing an identifier (or "trackID") used to identify each individual plant across years, survival in the next year, size in the next year, recruit status, and age. 

\begin{table}[h]
    \centering
    \resizebox{\textwidth}{!}{
    \small
    \begin{tabular}{p{0.22\linewidth}|p{0.75\linewidth}}
    \toprule
        \normal{ Function Name & Description} \\ [.8ex] \midrule 
        \rowcolor[gray]{.95} & \\
        \rowcolor[gray]{.95}trackSpp() & track plants through time to calculate growth and survival \\ 
        \rowcolor[gray]{.95} & \\
        & \\
        getNeighbors() & calculate a metric of competition for each individual \\ 
        & \\
        \rowcolor[gray]{.95} & \\
        \rowcolor[gray]{.95}groupByGenet() & group polygons into clones of a single genetic individual based on proximity \\
        \rowcolor[gray]{.95} & \\
        & \\
        checkDat() & ensure that data is in the correct format for use in other plantTracker functions \\
        & \\
        \rowcolor[gray]{.95} & \\
        \rowcolor[gray]{.95}drawQuadMap() & draw maps of quadrats, color coded either by species or by trackID\\
        \rowcolor[gray]{.95} & \\
        & \\
        getLambda() & calculate lambda (population growth rate for each species in each quadrat across all transitions in the data \\
        & \\
        \rowcolor[gray]{.95} & \\
        \rowcolor[gray]{.95}getRecruits() & calculate the number of new recruits in each year for each species \\
        \rowcolor[gray]{.95} & \\
        & \\
        getBasalAreas() & calculate the total basal area of each species in each quadrat in each year \\
        & \\
        \rowcolor[gray]{.95} & \\
        \rowcolor[gray]{.95}aggregateByGenet() & aggregate the output of the trackSpp() function so that each row  contains spatial data for all ramets of a genet (otherwise each ramet has it's own row in the output) \\ 
        \rowcolor[gray]{.95} & \\\midrule
    \end{tabular}}
    \caption{plantTracker functions}
    \label{tab:functions}
\end{table}

\subsection{Input Data}
plantTracker was designed for use with digitized chart-quadrat data, but can be used with any dataset where organisms are mapped and their species-identity recorded during at least two sequential time intervals. While it is possible to generate survival information if individuals were mapped at one point in space, it is best if individuals were mapped as polygons that reflect their basal area. This allows us to calculate growth and provides a more accurate estimation of competition. Even though plantTracker functions are flexible in that they can be used with differing types of map data, they require a specific data format to operate correctly. The input data.frame must be in the "sf" format, and each row must represent on observation in a given year. The "geometry" column can contain either a single polygon or a multipolygon if the observation is a single organism with multiple shoots or bunches. If the observation was mapped as a point, plantTracker requires it to be translated into a polygon, which can be a small circle of negligible area. In addition to spatial data, the input dataset must have columns containing information for the site of data collection, quadrat name, species name, and year of data collection for each observation. 

** have a figure showing an example chart-quadrat map**

\subsection{trackSpp() Function}
The trackSpp() function is the first of the two primary functions in plantTracker. It overlays maps of quadrats from consecutive years and assigns individual plants the same unique identifier if they overlap. It then uses these unique identifiers, which we call "trackIDs", to assign a value to each individual indicating survival to the next year , size in the next year ('NA' if survival = 0), a boolean value indicating whether it is a new recruit or not, and an age. The main arguments for trackSpp() are 1) 'dat', a spatial data.frame with the correct format and data described above, and 2) 'inv', a list in which each element is named after a quadrat in 'dat', and each element contains a vector of sampling years for the corresponding quadrat. 'dat' and 'inv' can contain data for as many quadrats and species as you'd like, as long as each has a unique name. There are four additional arguments that must be provided to trackSpp(), either globally or uniquely for each species, and has important implications for the resulting demographic data. 

The first is 'buff', which defines the distance that an individual can 'move' from year to year and still be considered the same individual. This distance, which must be specified in the same units as the spatial information in 'dat', can be used to account for small errors in mapping, or for true variation in where a perennial plant is resprouting each year. The value you use for the 'buff' argument is important for perennials with substantial basal area, but it is extremely important for plants with a very small basal area or plants that were originally mapped as points. If you use a 'buff' value of 0, then a very small plant must occur in exactly the same location in consecutive years to be considered the same individual. As a rule, larger values of 'buff' will slightly overestimate survival, while smaller values of 'buff' will slightly underestimate survival.   

The 'dorm' argument defines the maximum time period a plant is allowed to go 'dormant,' or disappear, from the quadrat map before reappearing and all along still be considered the same individual. For example, a plant of species 'A' is present at the coordinates (x,y) in 2000. No plants are near (x,y) in 2001 or 2002, but a plant of species 'A' is growing at (x,y) in 2003. If 'dorm' is set to 2, because this plant reappears at the same location after being gone for two years it is considered the same individual. Both the observations in 2000 and 2003 will receive the same trackID, and the observation in 2000 will receive a '1' in the survival column. If 'dorm' is set to 1 instead, the function will then conclude that the plant died after 2000, since the maximum dormancy of one year was exceeded. The observation in 2000 will receive a '0' in the survival column, and the 2003 observation will get a '1' in the recruit column and a new trackID. 'dorm' can be specified globally in the trackSpp() function call, or can be defined uniquely for each species. It is important to consider the biology of each species when specifying this argument. Allowing a year or two of dormancy for some forbs makes biological sense, but is unreasonable for species such as shrubs or large grasses. In addition to accounting for the biological reality of dormancy, the 'dorm' argument can also help account for plants being inadvertently missed in mapping. Allowing a 'dorm' value of 1 will also allow plants to survive through a year when sampling was skipped completely. Otherwise, all plants in the year after a break in sampling will be considered new recruits. Longer values of 'dorm' will slightly overestimate survival, while smaller values will slightly underestimate survival. 

The 'clonal' argument determines whether clonal or vegetative reproduction is allowed. If 'clonal' is TRUE, then a single genetic individual, a genet, can consist of several polygon observations, each of which is a ramet. The groupByGenet() function will be used within trackSpp() to group polygons together into genets based on their spatial proximity. Each ramet will be given the same trackID. The final argument required in trackSpp() is 'buffGenet', which determines how close together polygons must be for groupByGenet() to group them into one genet. If 'buffGenet' is equal to 1 cm, then a 1 cm buffer is drawn around each polygon, and if any two touch they are considered the same genet--so 'buffGenet' is 1/2 of the maximum distance allowed between ramets. If 'clonal' is FALSE, then vegetative reproduction is not allowed and every polygon observation is considered a distinct genetic individual. If 'clonal' = FALSE, then a value does not need to be supplied to the 'buffGenet' argument. As with the other required arguments in trackSpp(), both 'clonal' and 'buffGenet' can be defined globally for all species, or on a species-specific basis.  

** nuts and bolts description of what the trackSpp function is actually doing. can't decide if I should have a sort of dichotomous key? Or a diagram of the decision points? Or both? Right now refer to the trackSpp() vignette for a detailed description of the process** 

\subsection{getNeighbors() Function}
The getNeighbors() function is the other primary worker function in plantTracker. This function calculated a metric of competition for each individual in the dataset it each year, and can be customized based on the data type and desired outcome. It is often useful in demographic analyses to have some idea of the competition (or facilitation) that an individual organism is dealing with. Interactions between individuals can have a profound impact on whether an organism survives and grows. Spatial datasets of plant occurrence allow us to generate an estimate of the interactions an individual plant has with other plants by determining how many other individuals surround each plant. While this isn't a direct measure of competition or facilitation, it gives us an estimate that we can include in demographic models.

The `getNeighbors()` function in PlantTracker calculates this competition 
estimate for each unique individual a mapped dataset. A user-specified buffer is drawn around each individual, and then quantifies the number of other 
plants within this buffer. Specifying this buffer is another point where knowing the biology of the species present in the map dataset is important. A larger buffer will increase the competition metric, so it is important to use a buffer size that estimates the actual radius within which competitors are significantly impacting the focal individual. 

'getNeighbors()' requires a dataset where each genet is represented by only one row in the spatial data.frame, and each genet has a trackID that is consistent across years. This data can come directly from a trackSpp() if the 'aggregateByGenet' argument is set to TRUE. `getNeighbors()` has several options that allow the user to customize how the competition metric is calculated. First, the user can decide how the function 'counts' other plants inside the buffer zone around the focal individual. In option 1,  The function will calculate a tally of the number of genets inside the buffer zone. In option 2, the function will calculate the proportion of the area of the buffer zone that is occupied by other plants. The second option is perhaps more likely to be biologically accurate if a basal area was mapped for every individual. However, if individuals were mapped originally as points, it is best to use the first approach so as not to underestimate the effect of plants that were mapped with what amounts to a tiny basal area. The second user-control on the output of 'getNeighbors' determines whether the function will calculate a metric of intraspecific competition and only consider other plants in the buffer zone that are the same species as the focal individual,  or whether it will calculate interspecific competition and consider all other plants in the buffer zone, regardless of species. The data.frame returned by 'getNeighbors' is identical to the input data.frame, but includes an additional column called 'neighbors'. This column contains either a count of the number of individuals within each focal individual's buffer for the "count" method, or a proportion of the buffer area that is occupied by other individuals for the "area" method. 

\subsection{Additional Functions}
** not sure if I need to go into detail to describe the other functions in the package...* 

\section{Proof of Concept}
\subsection{Three consecutive years of mapped \textit{Oenothera coloradensis} data}
** description of data type **

** some sort of summary statistics of how the functions performed

\subsection{Large-scale repeated measurements of tree basal area}
** description of data type **

** some sort of summary statistics of how the functions performed

\section{Conclusion}
plantTracker provides a suite of user-friendly tools that make it possible to easily generate demographic data from maps of plants that do not include unique individual identifiers across years. These tools expand the number of demographic datasets at our disposal by allowing us to translate both historical and contemporary map datasets into information about growth, survival, and recruitment. Robust and long-term estimation of vital rates is critical to driving forward many disciplines in ecology, and we hope users will find plantTracker helpful for generating data to fill gaps in our ecological understanding.  

\bibliography{references.bib}

\end{document}
